\begin{problem}{Koolhash}
{stdin}{stdout}
{2 seconds}{}{}

U of I's Chancellor - Robert Jones - has thought of a new hashing function to store administrator records, and he's enlisted you to help him build it! His hash function $\phi(n)$ will output a single integer, and needs the following pieces of information: $l$, $r$, and $N$, where $l$ represents the leftmost bit of  $n$, $r$ represents the rightmost bit of $n$, and $N$ represents the number of 1-bits in $n$. \\ \\
There aren't too many administrator records, so for now let's assume an unsigned 32-bit number system. Your job is to provide Chancellor Jones with the three components of his hashing function given the number of records $k$, and the unique identifier of each record $n$. And yes, please forget about the absurdity of his hashing algorithm. Unfortunately, our Chancellor has never taken a formal CS class.

\InputFile
The first line contains the number of records $k$, $1 \leq k \leq 500$. \\ \\
Each of the next $k$ lines contains an integer representing a record identifier $n$, $0 \leq n < 2^{31}$.\\ \\
You can assume all inputs are given in decimal, and you need to convert them into 32-bits unsigned integers to get $l$, $r$ and $N$ for each $n$.

\OutputFile
On each line, print three space-separated integers for every input record, denoting the leftmost bit $l$ of $n$, the rightmost bit $r$ of $n$, and number of 1-bits bits $N$ of $n$, respectively. \\ \\

For further clarification, see the explanation provided for the first example below.
\Examples

\begin{example}
\exmp{
1
8
}{%
0 0 1
}%
\end{example}

\begin{example}
\exmp{
3
45
68
23
}{%
0 1 4
0 0 2
0 1 4
}%
\end{example} \\
\Explanation
For the first example, $8$ is represented as $0x00000008$ (written in hexadecimal for simplicity). Thus, the left most bit is 0, the right most bit is 0, and the number of 1-bits is 1.
\end{problem}
