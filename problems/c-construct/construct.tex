\begin{problem}{Construct Underground System}
{stdin}{stdout}
{2 seconds}{}{}

The United Institute for Underground Construction(UIUC) is a famous company that is dedicated to constructing the best underground(subway) system. But how good can an underground system be without vending machines at the stations? As an insightful employee, \pittoresque decided to offer a plan of building some vending machines at some stations. But there are some details to consider:
\begin{itemize}
    \item As the stations are crowded, there is only enough space to build one machine at each station.
    \item The supply for all vending machines comes from a single manufacturing factory, which is a constant number. And the total amount of drinks sold should not exceed this number,
    \item For each station with a vending machine, a fixed amount of people will buy drinks at a certain price. 
\end{itemize}

To be specific, there are $n$ stations numbered from $1$ to $n$. At the $i$th station, you can place a vending machine such that exactly $p_i$ people come and buy drinks, and each drink will provide a revenue of $v_i$. Since \pittoresque is in a small town(of course, not many people lives in a corn field) the total number of people that buy drinks across all the stations is $\sum_{i=1}^n p_i = P \leq 2 * 10^5$. Please help \pittoresque calculate the maximum revenue he can get by constructing the stations. 

\InputFile

The first line contains two integers $n$ and $W$ ($1 \le n \le 10^5$, $1 \le n \le 2 * 10^5$), the number of stations and the number of number of drinks supplied by the manufacturing company.

In the following $n$ lines, the $i$-th line contains two integers $p_i$ and $v_i$($1 \le p_i \le 2 * 10^5$, $1 \le v_i \le 10^9$), the number of people buying drinks at each station and the revenue for each drink at that station.

It is guaranteed that $\sum_i p_i \leq 2 * 10^5$.

\OutputFile

Output the maximum revenue among all ways to place the vending machines.

\Examples

\begin{example}
\exmp{
3 10
1 2
1 10
11 100000
}{%
12
}%
\end{example}

\begin{example}
\exmp{
4 10
3 1
5 5
3 3
4 6
}{%
34
}%
\end{example}



\end{problem}