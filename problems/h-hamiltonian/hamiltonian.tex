\begin{problem}{Hamiltonian Farm}
{stdin}{stdout}
{3 seconds}{256 MB}{}

Last year, your team failed on helping the Codefalser \Suzukaze  AK the problemset. Therefore, \Suzukaze decided to retire from competitive programming and became a farmer since he wants to own a farm as \pittoresque does. \Suzukaze is an orange-lover so he decided to plant only orange trees in his farm in spring. Winter is coming next week! \Suzukaze is planning to walk inside his farm to harvest his favorite fruit. However, as a forgetful farmer, \Suzukaze loses his memory about the configuration of his farm. Fortunately, as a careful programmer, \Suzukaze stored the configuration of his farm in the computer as a function in case of he gets into this kind of desperate situation.
\par
The farm can be modelled as a graph with $n$ vertices where vertices are orange trees, and the edges in the graph can be derived from the function $f$:

$$f(i,j)=
\begin{cases}
0& \text{$i=j$}\\
((ip)^{jq}\ mod\ (10^{9}+7))\ mod\ 2& \text{$i<j$}\\
1-f(j,i)& \text{$i>j$}
\end{cases}$$

where $i$ and $j$ are the indices of the vertices ($1\leq i,j\leq n$), $p$ and $q$ are non-negative integers less than $10^{9}+7$. If $f(i,j)=1$, there is a directed edge from $i$ to $j$.

\par
As a lazy farmer, \Suzukaze wants to find a path that can visit each orange tree exactly once. Can you help him find this path in compensation for your failure last year?

\InputFile

The first line contains three integers $n$,\ $p$ and $q$ ($1 \leq n \leq 10^{5}$, $0 \leq p,q < 10^{9}+7,$ $p$ and $q$ can't be 0 at the same time), the number of orange trees and the parameters of the function. You may assume that the orange trees have indices $1, ..., n$.

\OutputFile

If the path exists, output the vertices on the path from the beginning vertex to the end vertex as the example shows. Any of the path that satisfies \Suzukaze 's demand will be accepted. Otherwise, output -1, which means that you fail on \Suzukaze 's request again.

\Examples

\begin{example}
\exmp{
6 1 1
}{%
1
3
5
6
4
2
}%
\end{example}

\Explanation
The path $1\xrightarrow{}3\xrightarrow{}5\xrightarrow{}6\xrightarrow{}4\xrightarrow{}2$ satisfies \Suzukaze's demand in the example.

\end{problem}
