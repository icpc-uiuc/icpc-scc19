
\begin{problem}{Fruit on the Tree}
{stdin}{stdout}
{2 second}{}{}

``Triangoes'', a new type of fruit, are in triangular shapes and taste like mangoes. However, nobody in the world has ever seen ``triangoes'' since they always grow implicitly on the tree and hide in the triangles. Formally, a ``triango'' tree is an undirected tree with weighted edges and a ``triango'' is a set of three vertices on the ``triango'' tree such that the lengths of the three simple paths between each pair of these three vertices satisfy the triangle inequality; that is to say, they form a triangle.

After a long expedition, \Suzukaze has eventually found a ``triango'' tree in his house. He needs your help to count the number of ``triangoes'' on the ``triango'' tree. Can you help him?

\InputFile

The first line contains an integer $n$ ($1 \le n \le 10^5$), the number of vertices on the ``triango'' tree. The vertices on the ``triango'' tree are labeled from 1 to $n$.

In the following $n-1$ lines, each of the lines contains 3 integers $u, v, w$ ($1 \le u,v \le n, u \neq v, 1 \le w \le 10^5$), which means that there is an edge with weight $w$ connecting vertex $u$ and vertex $v$.

It's guaranteed that the input data is a single connected tree.

\OutputFile

Output an integer - the number of ``triangoes'' on the ``triango'' tree.

\Examples

\begin{example}
\exmp{
7
1 2 1
1 3 1
2 4 1
2 5 1
3 6 1
3 7 1
}{%
8
}%
\end{example}

\end{problem}
