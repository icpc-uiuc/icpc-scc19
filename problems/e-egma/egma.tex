\begin{problem}{Egma Game}
{stdin}{stdout}
{2 seconds}{256 MB}{}

As we all know, \TiChuot is \textit{one of} the greatest professional gamers of all time. And, similar to other great gamers, he loves games,
especially nim games. Today, he just found out an online nim game - Egma. As other nim games, Egma requires proficiency in computing mex values
in order to master it. \TiChuot understands that, just like millions of other games he mastered, Egma requires practicing. This is where his
best friend, \T, comes in to help.\\

\T has prepared a drill for \TiChuot's practice. Initially, \TiChuot is given an array of size $n$ of nonnegative integers $a_1, a_2, ..., a_n$.
Then, \T will give \TiChuot $q$ queries each consists of two numbers $l, r$ $(1 \leq l \leq r \leq n)$ asking for the mex of $\{a_l, a_{l + 1},
..., a_r\}$. Of course, \TiChuot finished this drill easily. However, he thinks that this challenge can improve, not only his mex-computing skill,
but also his programming skill. Do you also want to give this challenge a try?

\textbf{Note}: Mex value of a set of nonnegative integers is defined to be the minimum nonnegative integer that does not belong to the set.

\InputFile
The first line contains an integer $n$ ($1 \leq n \leq 5\times10^5$), the length of the initial array. The second line contains $n$ integers $a_1,
a_2, ..., a_n$ ($0 \leq a_i \leq 10^9$). The third line contains an integers $q$ ($1 \leq q \leq 5\times10^5$), the number of queries. Each of the
next $q$ lines contain a pair $l, r$ $(1 \leq l \leq r \leq n)$ describing a query.

\OutputFile

For each query, output on one line the answer to such query.

\Examples

\begin{example}
\exmp{
5
1 2 3 0 5
2
1 3
1 4
}{%
0
4
}%
\end{example}

\end{problem}
