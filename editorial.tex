\documentclass{article}

\usepackage{amsmath}
\usepackage{amsfonts}
\usepackage{mathtools}
\DeclarePairedDelimiter\ceil{\lceil}{\rceil}
\DeclarePairedDelimiter\floor{\lfloor}{\rfloor}

\title{UIUC ICPC Spring Coding Contest 2019 \\ Editorial}
\date{}

\begin{document}

\maketitle

\section*{A - AK the Problems}



\section*{B - Bigram Language Model}

For each query $(s,t)$, the nominator is the number of co-occurrences of $(s,t)$ in the corpus, and the denominator is the number of occurrences of $s$ as non-terminal word in a sentence. These values can be pre-processed by a single pass over the corpus. 

The tricky part is that the number of distinct words appearing in the corpus $w$ may be up to $10^5$, so it is impossible to explicitly construct the co-occurrence matrix of size $w \times w$. The observation is that the sum of all elements in this matrix is bounded by the size of corpus, so we can use some sparse-matrix representation techniques (e.g. only recording which entries are non-zero and the values they hold). 

\section*{C - Construct Underground System}



\section*{D - Diameter}



\section*{F - Fruit on the Tree}



\section*{G - Greenberg Mass Comparison}

The core of this problem is asking you how many ways are there to partition a set of $n$ elements into disjoint, non-empty subsets. If you know Bell number, that's exactly what we are asking for (but you still have to do the programming). If not, not a problem! Let's work out a recurrence formula for it. 

Let $B_n$ denote the answer for $n$. Consider the number of elements \textbf{NOT} in the subset containing element $1$. If there are $k$ such elements, then there are $\binom{n-1}{k}$ ways to choose these elements, and we can just ignore the subset containing $1$ and do the partition on these $k$ elements. Therefore, 
\begin{align*}
	B_n &= \sum_{k=0}^{n-1}{\binom{n-1}{k} B_k}
\end{align*}
with base case $B_0 = 1$. 

Notice that Bell numbers can get very large quickly, so coding this problem in Python or Java will save you a lot of trouble. 

\section*{J - Juicy}



\end{document}